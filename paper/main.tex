\documentclass[conference]{IEEEtran}
\IEEEoverridecommandlockouts
% The preceding line is only needed to identify funding in the first footnote. If that is unneeded, please comment it out.
\usepackage{cite}
\usepackage{amsmath,amssymb,amsfonts}
\usepackage{algorithmic}
\usepackage{graphicx}
\usepackage{textcomp}
\usepackage{xcolor}
\usepackage{xspace}
\usepackage{paralist}

\def\BibTeX{{\rm B\kern-.05em{\sc i\kern-.025em b}\kern-.08em
    T\kern-.1667em\lower.7ex\hbox{E}\kern-.125emX}}
\begin{document}

\newcommand{\tool}{\textsc{SALVO}\xspace}
\newboolean{showcomments}
\setboolean{showcomments}{true}         
%\setboolean{showcomments}{false} 
\ifthenelse{\boolean{showcomments}}
  {\newcommand{\nb}[2]{
  \fbox{\bfseries\sffamily\scriptsize#1}
     {\sf\small$\blacktriangleright$\textit{\textcolor{red}{#2}}$\blacktriangleleft$}
   }
  }
  {\newcommand{\nb}[2]{}
   \newcommand{\cvsversion}{}
  }

\newcommand\alessio[1]{\nb{Alessio}{#1}} 
\newcommand\vuong[1]{\nb{Vuong}{#1}} 
\newcommand\stefan[1]{\nb{Stefan}{#1}} 

\title{\tool: Automated Generation of Diversified Tests from Existing Maps
\alessio{Title is not final}
}

\author{\IEEEauthorblockN{Alessio Gambi}
\IEEEauthorblockA{\textit{Passau University} \\
Passau, Germany \\
alessio.gambi@uni-passau.de}
\and
\IEEEauthorblockN{Vuong Nguyen}
\IEEEauthorblockA{\textit{Passau University} \\
Passau, Germany \\
nguyen58@ads.uni-passau.de}
\and
\IEEEauthorblockN{Stefan Huber}
\IEEEauthorblockA{\textit{Passau University} \\
Passau, Germany \\
stefan.huber.niedling@outlook.com}
}

\maketitle

% Another thing: as the deadline for the final phase of the challenge is fast approaching, we would like to encourage you to submit a paper describing the results of your (team)work.
% The papers are accepted as part of the 2021 IEEE AI Testing Conference and should be submitted via EasyChair (make sure that you select the "Only for IEEE AV AI Test Challenge 2021" option). The submissions are expected to be 6-10 pages long and should present an overview of the scenarios created during phases 1 and 2 of the challenge as well as results of test execution accompanied by the conclusion. The format of the papers is described on the main conference CFP page.  
% The deadline for paper submissions is July 23. We will be sending notifications of acceptance during the week of August 9. 

\begin{abstract}
Testing self-driving car software using photo-realistic and physically accurate simulators is becoming a widely adopted practice. Simulators enable the automated execution of many tests that possibly implement safety-critical driving scenarios that are unlikely or simply too dangerous to put into practice. However, despite the speed-up achieved by simulation-based testing, fundamental challenges are yet to be addressed before achieving cost-effective testing of self-driving cars.

In this paper, we present \tool, an approach to automated self-driving car software testing that addresses the following challenges:
\begin{inparaenum}[(i)]
\item systematically identifying relevant driving scenarios among the myriad of potential driving scenarios that can be defined over existing maps;
\item selecting driving scenarios that minimize testing effort while ensuring quantifiably different behaviors, 
and \item 
automatically generating test cases that implement those relevant and diversified driving scenarios.
\end{inparaenum}

We demonstrated \tool in the context of the 2021 IEEE Test AI Challenge
by testing state-of-art self-driving car software using an industrial driving simulator.
In a matter of minutes, \tool analyzed five maps of different complexity (from $2$ to more than $15$ junctions),
identified more than $1100$ relevant driving scenarios in them, and selected on average half of
them for the execution.
%
The evaluation results showed that tests cases generated by \tool from the selected driving scenarios
resulted in quantifiably different trajectories of the test subject and exposed issues in it.

\end{abstract}

%CubeTown
%1. 10 paths
%2. distance 1.9: 8 paths
%3. 10 cells: 5 paths
%4. 2 intersections
%BorregasAve
%1. 28 paths
%2. distance 1.9: 20 paths
%3. 10 cells: 16 paths
%4. 2 intersections
%Shalun
%1. 140 paths
%2. distance 1.9: 110 paths
%3. 10 cells: 33 paths
%4. 14 intersections
%SanFrancisco
%1. 806 paths
%2. distance 1.9: 535 paths
%3. 10 cells: 52 paths
%4. 89 intersections
%Gomentum
%1. 126 paths
%2. distance 1.9: 110 paths
%3. 10 cells: 25 paths
%4. 25 intersections

\begin{IEEEkeywords}
\alessio{todo}
\end{IEEEkeywords}

\section{Introduction}
\alessio{todo}


%\section*{Acknowledgment}
%
%The preferred spelling of the word ``acknowledgment'' in America is without 
%an ``e'' after the ``g''. Avoid the stilted expression ``one of us (R. B. 
%G.) thanks $\ldots$''. Instead, try ``R. B. G. thanks$\ldots$''. Put sponsor 
%acknowledgments in the unnumbered footnote on the first page.

% References
\bibliographystyle{IEEEtran}
\bibliography{biblio}

\end{document}
